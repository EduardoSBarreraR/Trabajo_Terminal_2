\chapter{Conclusiones}

El seguimiento de la metodolog�a permiti� cumplir satisfactoriamente los objetivos establecidos para Trabajo Terminal 1, ya que funcion� como gu�a a lo largo del proceso de dise�o desde el planteamiento de las necesidades hasta la secuencia de integraci�n, en la cual se demuestra que los m�dulos trabajan de manera arm�nica. Sin embargo, es importante tomar en cuenta que hay puntos pendientes por desarrollar, como lo son la din�mica del robot, planeaci�n de trayectoria y el control de los actuadores.

Por otro lado, medir solamente el peso de la piedra no permite conocer las caracter�sticas del entorno en el que se encuentra como requer�an los objetivos, sin embargo, el instrumental requerido para realizar estudios que as� lo permitan son prohibitivamente costosos, e.g. espectrograf�a, por lo que su implementaci�n queda fuera de la viabilidad del proyecto. 

Sin embargo, se considera que el dise�o alcanz� un grado de madurez aceptable para comenzar la implementaci�n en Trabajo Terminal 2.
