\chapter{Conclusiones}

Conforme al an�lisis de ingenier�a realizado se puede concluir que el funcionamiento del sistema es correcto a nivel modular, aunque presenta fallos en la integraci�n. Esto se puede atribuir a diversas causas, por ejemplo, el uso cont�nuo del manipulador provoca holgura mec�nica entre sus componentes, lo cual es dificil de corregir sin la infraestructura adecuada. A�n as�, el funcionamiento general del sistema se encuentra dentro de par�metros aceptables. 

Tomando en cuenta lo anterior y promediando los porcentajes calculados para los objetivos particulares, se determina que el objetivo general se satisface al 86.97\%.

\section{Trabajo a futuro}
\label{sec:TrabajoFuturo}

El uso de encoders incrementales hace necesario conocer la posici�n inicial, que debe ser calibrada manualmente, lo que afecta la repetibilidad del sistema. Esto se puede solucionar utilizando encoders absolutos, que eliminan la necesidad de tener una posici�n conocida. 

La masa no da una descripci�n completa de las caracter�sticas del entorno, por lo que se requiere mejorar el sistema de laboratorio para obtener m�s informaci�n de la muestra recogida, como la composici�n qu�mica a trav�s de an�lisis espectrogr�fico.

Ocurren ca�das repentinas del manipulador asociadas a la configuraci�n del robot, lo que puede solucionarse al implementar un control m�s robusto, que sea capaz de reaccionar r�pidamente a cambios en la distribuci�n del peso de la muestra tomada.

El sistema es vulnerable a las condiciones del entorno, como polvo, agua, temperatura y el campo electromagn�tico, por lo que se pueden aislar los componentes del robot de las condiciones del entorno, tales como cables, bandas y motores, a fin de protegerlo.

El efector requiere que la muestra estuviera centrada para tener un mejor agarre, por lo que un ligero desv�o provocaba que los dedos cerraran de manera dispareja, y al momento de levantarla el dedo que no ejerc�a la presi�n necesaria, permitiendo que la muestra cayera, por lo que se requiere un an�lisis en la longitud de las falanges subactuadas, adem�s de mejorar el rango de movimiento de las mismas, para permitir al efector abrirse m�s.

No existe una manera de restringir el movimiento del manipulador, incrementando el riesgo de colisiones con el sistema estructural y/o con los mismos elementos del manipulador. Se pueden colocar sensores (fin de carrera, capacitivos o inductivos) en puntos estrat�gicos del sistema, para as� mantener su integridad f�sica.

La orientaci�n paralela del Kinect con la superficie limita el campo de visi�n, el cual podr�a ampliarse y abarcar todo el espacio de trabajo del robot si se gira, calculando la matriz de rotaci�n sobre los ejes principales para trasladar las coordenadas obtenidas con respecto del marco referencial del Kinect al del robot.

En la interfaz de usuario, el operador solo es capaz de ver la imagen de profundidad, la cual puede ser dificil de interpretar, por lo que es posible mostrar la imagen RGB proporcionada por el Kinect implementando una funci�n de correspondencia entre esta con la imagen de profundidad.

La holgura mec�nica entre los componentes representa un problema que disminuye la precisi�n del robot (e.g. los opresores que sujetan las poleas da�an el eje), lo que podr�a mejorarse al cambiar el m�todo de sujeci�n.

Durante las pruebas se hizo patente una alta probabilidad de que el efector golpee el objeto durante la aproximaci�n al mismo, este problema puede ser solventado implementando una generaci�n de trayectoria en el espacio de trabajo que se aproxime verticalmente a la muestra objetivo. 