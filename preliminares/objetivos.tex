\chapter{Objetivos}

\section*{Objetivo general}
Dise�ar y construir un sistema rob�tico de recolecci�n de muestras geol�gicas para un rover
de exploraci�n, que sea capaz de obtener piedras del suelo para su posterior an�lisis y env�o
de datos a una estaci�n.


\section*{Objetivos Particulares.}


\begin{enumerate}
\item Trabajo Terminal I
	\begin{itemize}[noitemsep,nolistsep]
	
	\item Dise�ar un m�dulo a manera de efector final 		que sea capaz de recoger muestras del
	suelo para su manipulaci�n.
	
	\item Dise�ar un sistema rob�tico de 6 grados de 		libertad que posicione y oriente el efector
	para transportar la muestra.

	\item Dise�ar un m�dulo de percepci�n que le 			permita al sistema reconocer el ambiente
	para crear un marco de referencia en el espacio.

	\item Dise�ar un sistema de procesamiento y 			comunicaci�n que permita el an�lisis de las
	muestras para determinar las caracter�sticas del 		entorno, adem�s de permitir la
	transmisi�n y recepci�n de datos. 

	\item Integrar los m�dulos de manera computacional 	para que trabajen en conjunto.

	\item Validar el sistema mecatr�nico mediante 			simulaciones con el objetivo de comprobar
	que los par�metros obtenidos cumplen su funci�n.
	
	\end{itemize}

\item Trabajo Terminal II
	\begin{itemize}[noitemsep,nolistsep]
	
	\item Implementar el m�dulo de efector final que 		sea capaz de recoger muestras del suelo
	para su manipulaci�n.
	
	\item Implementar un sistema rob�tico de 6 grados 		de libertad que posicione y oriente el
	efector para transportar la muestra.

	\item Implementar un m�dulo de percepci�n que le 		permita al sistema reconocer el
	ambiente para crear un marco de referencia en el 		espacio, as� como identificar la
	geometr�a de la muestra.

	\item Implementar un sistema de procesamiento y 		comunicaci�n que permita el an�lisis de
	las muestras para determinar las caracter�sticas 		del entorno, adem�s de permitir la
	transmisi�n y recepci�n de datos.

	\item Realizar pruebas y ajustes necesarios para 		garantizar que cada m�dulo trabaja
	correctamente de manera independiente.

	\item Integrar secuencialmente los m�dulos para 		consolidar el sistema mecatr�nico.

	\item Verificar que todos los m�dulos del sistema 		mecatr�nico trabajen arm�nicamente para
	posteriormente realizar pruebas de recolecci�n de 		muestras en un ambiente an�logo
	que nos permitan comprobar que el sistema cumple 		con la funci�n principal.
	
	\end{itemize}
\end{enumerate} 