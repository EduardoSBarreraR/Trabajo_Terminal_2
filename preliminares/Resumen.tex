\chapter{Resumen}




\textit{\textbf{Abstract-}} Geologic data acquisition of remote solar system bodies, is only feasible by
means of robots, which are tested in scenarios that recreate conditions in which they want to
be operated. To achieve a soil analysis, tests require \textit{in situ} samples to be taken. Usually, exploration is carried out by
\textit{rover}-type mobile robots, which besides the motion system, consist of analytical instruments
and end effectors to obtain the required samples. Due to the sample acquisition is already a
complex problem, in this document the design of a mechatronic system is developed, which by means of a 6 degrees-of-freedom manipulator and an 3 finger gripper-like end effector, picks a sample, identified by computer vision, from the ground and places it in a specified place for its analysis, and sends the result to a computer for its display.

\newpage

\textit{\textbf{Resumen-}} La obtenci�n de datos geol�gicos de cuerpos distantes del sistema solar, s�lo es viable mediante robots, los cuales son probados en escenarios que recrean condiciones
en las que quieren ser operados. Es de gran inter�s cient�fico el an�lisis de terreno, para lo
cual se pueden realizar pruebas que requieren la obtenci�n de muestras \textit{in situ}.
Generalmente, la exploraci�n se lleva a cabo mediante robots m�viles de tipo \textit{rover}, los
cuales incluyen adem�s del sistema de locomoci�n, instrumental de an�lisis y efectores para
obtener las muestras de inter�s. Dado que la obtenci�n de la muestra presenta un problema
complejo, en este documento se desarrolla el dise�o de un sistema mecatr�nico que
mediante un manipulador de 6 grados de libertad y un efector de tipo gripper de 3 dedos, recoja una muestra del suelo identificada mediante visi�n por computador y la ubique en un lugar espec�fico para su an�lisis, enviando el resultado obtenido a un ordenador para su visualizaci�n.


\textbf{Palabras Clave:} Sistema rob�tico, exploraci�n, an�lisis de muestras, brazo antropom�rfico, \textit{rover}. 