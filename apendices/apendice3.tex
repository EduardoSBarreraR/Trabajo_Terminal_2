\chapter{Ap�ndice 2: Simulaci�n cinem�tica del robot}
\setcounter{section}{0}
\section{C�digo de simulaci�n de cinem�tica de robots utilizando el toolbox ``Robotics'' para MATLAB \textregistered}
\label{Ap:CodigoRoboCorke}
\begin{lstlisting}[language=Matlab]
clc, clear all,close all
%% Configuraci�n del robot

O1 = 5;
O2 = 5;
L1 = 32;
L2 = 24;
L3 = 10;

L(1)=Link('d',O1,'a',O2,'alpha',pi/2);%par�metros de la primer junta y eslab�n
L(2)=Link('d',0,'a',L1,'alpha',0);%par�metros de la segunda junta y eslab�n
L(3)=Link('d',0,'a',L2,'alpha',pi/2);
L(4)=Link('d',0,'a',0,'alpha',-pi/2,'offset',-pi/2);%%%%%%%% Link 
L(5)=Link('d',0,'a',0,'alpha',pi/2,'offset',pi);%%%%%%%% Link 
L(6)=Link('d',0,'a',0,'alpha',pi/2,'offset',pi);%%%%%%%% Link 
L(7)=Link('d',L3,'a',0,'alpha',0);%%%%%%%% Link 


qli=[-pi pi;-pi pi;-pi pi;0 0;-pi pi;-pi pi;-pi pi];%definici�n de los l�mites de las juntas 
qz=[0 0 0 0 ];%matriz con vectores articulares rengl�n
%%qfinal=[-pi/2 0 pi/2];
antropom=SerialLink(L,'name','antrop','qlim',qli); %construcci�n del objeto polar, 
antropom.teach;
%= {'workspace', [-6 6 -6 6 -6 6]}; %opcion del m�todo plot,

%% Trazado de la trayectoria
%path = [ -4 1 2; -2.82 1 4.82; 0 1 6; 2.82 1 4.82; 4 1 2]; % secuencia de puntos en R^3
%plot3(path(:,1), path(:,2), path(:,3), 'color', 'k', 'LineWidth', 2)
%grid on
%hold on

%% Seguimiento de trayectoria con el robot
%d=interp_mia(path,50);%trayectoria en R^3 y tama�o de paso
%q=iantrop(d); %mapeo de la trayectoria de d \in R^3 a q \in R^n
%qal=iantrop(path)
%q=interp_mia(qal,50)
%antropom.plot(q) % grafica del seguimiento

hold off
\end{lstlisting}